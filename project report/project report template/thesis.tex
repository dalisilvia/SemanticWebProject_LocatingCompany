% do not change these two lines (this is a hard requirement
% there is one exception: you might replace oneside by twoside in case you deliver 
% the printed version in the accordant format
\documentclass[11pt,titlepage,oneside,openany]{book}
\usepackage{times}


\usepackage{graphicx}
\usepackage{latexsym}
\usepackage{amsmath}
\usepackage{amssymb}

\usepackage{ntheorem}

% \usepackage{paralist}
\usepackage{tabularx}
\usepackage{url}

% this packaes are useful for nice algorithms
\usepackage{algorithm}
\usepackage{algorithmic}

% well, when your work is concerned with definitions, proposition and so on, we suggest this
% feel free to add Corrolary, Theorem or whatever you need
\newtheorem{definition}{Definition}
\newtheorem{proposition}{Proposition}


% its always useful to have some shortcuts (some are specific for algorithms
% if you do not like your formating you can change it here (instead of scanning through the whole text)
\renewcommand{\algorithmiccomment}[1]{\ensuremath{\rhd} \textit{#1}}
\def\MYCALL#1#2{{\small\textsc{#1}}(\textup{#2})}
\def\MYSET#1{\scshape{#1}}
\def\MYAND{\textbf{ and }}
\def\MYOR{\textbf{ or }}
\def\MYNOT{\textbf{ not }}
\def\MYTHROW{\textbf{ throw }}
\def\MYBREAK{\textbf{break }}
\def\MYEXCEPT#1{\scshape{#1}}
\def\MYTO{\textbf{ to }}
\def\MYNIL{\textsc{Nil}}
\def\MYUNKNOWN{ unknown }
% simple stuff (not all of this is used in this examples thesis
\def\INT{{\mathcal I}} % interpretation
\def\ONT{{\mathcal O}} % ontology
\def\SEM{{\mathcal S}} % alignment semantic
\def\ALI{{\mathcal A}} % alignment
\def\USE{{\mathcal U}} % set of unsatisfiable entities
\def\CON{{\mathcal C}} % conflict set
\def\DIA{\Delta} % diagnosis
% mups and mips
\def\MUP{{\mathcal M}} % ontology
\def\MIP{{\mathcal M}} % ontology
% distributed and local entities
\newcommand{\cc}[2]{\mathit{#1}\hspace{-1pt} \# \hspace{-1pt} \mathit{#2}}
\newcommand{\cx}[1]{\mathit{#1}}
% complex stuff
\def\MER#1#2#3#4{#1 \cup_{#3}^{#2} #4} % merged ontology
\def\MUPALL#1#2#3#4#5{\textit{MUPS}_{#1}\left(#2, #3, #4, #5\right)} % the set of all mups for some concept
\def\MIPALL#1#2{\textit{MIPS}_{#1}\left(#2\right)} % the set of all mips


\begin{document}

\pagenumbering{roman}
% lets go for the title page, something like this should be okay
\begin{titlepage}
	\vspace*{2cm}
  \begin{center}
   {\Huge Finding Generic Drugs \\}
	\vspace*{0.5cm}
   {\large  interlinking DBPedia with FreeDB based on drug brands \\}
   \vspace{2cm} 
   \vspace{2cm}
   {written by\\
    Markus Dietsche 1513384 \\
    Dandan Li 1486051\\
   }
   \vspace{1cm} 
   {submitted to the\\
    Data and Web Science Group \\
    Prof.\ Dr.\ Paulheim\\
    University of Mannheim\\} \vspace{2cm}
   {December 2015}
  \end{center}
\end{titlepage} 

% no lets make some add some table of contents
\tableofcontents
\newpage

%\listofalgorithms %TODO: we dont need it

%\listoffigures %TODO: we dont need it

%\listoftables %TODO: we dont need it

% evntuelly you might add something like this
% \listtheorems{definition}
% \listtheorems{proposition}

\newpage


% okay, start new numbering ... here is where it really starts
\pagenumbering{arabic}

% Legend:
% TODO: 		general todo, doesnt matter who takes care of it
% TODO M: 	todo for Markus
% TODO S: 	todo for Silvia


\chapter{Application domain and goals}
\label{cha:domain}
This documents describes the development of a generics search engine, based on the publicly available knowledge bases DBPedia\footnote{\url{www.dbpedia.org}} and FreeBase\footnote{\url{www.freebase.com}}. Furthermore it discusses the limitations of the corresponding TRL6\footnote{Technology Readiness Level \url{http://resources.sei.cmu.edu/library/asset-view.cfm?assetID=5835}} prototype and contains proposals for further development.

\paragraph{Problem} \label{sec:problem}
% Which user information needs are addressed
Currently people are forced to rely on expert opinion to find a substitute, e.g a cheaper alternative, to their medication. Since many experts, like doctors or pharmacists, are granted benefits from pharmaceutical companies their judgment is often assorted to be biased.
Furthermore drugs often have fantasy names, due to marketing reasons. This is in particular the case for over-the-counter drugs. Therefore a comparison is only possible based ingredients.
In case of drugs without patent protection this potentially results in filtering thousands of different drugs by their ingredients. Needless to say that even well informed an unbiased expert will not know all generics by heart.



\paragraph{Solution} 
%\paragraph{Target group}\label{sec:users}
% Which users are targeted?
The search engine is designed for private individuals, without any medical expert knowledge.
\label{sec:solution} % Which user problems are solved?
 By entering a drug brand name or active substance the prototype will return a list of identical substitute drugs and their corresponding drug manufacturers.


% \section{Demand Analysis} %TODO S: what is this section?



\chapter{Datasets}
\label{cha:datasets}

\paragraph{Datasets used}
\label{sec:datasets_used}
% which datasets does the application use

\subparagraph{DBPedia}
\label{dbpedia}
% TODO S: describe DBPedia (important: dont copy and paste + cite correctly!)

\subparagraph{FreeBase}
\label{freebase}
% TODO S: describe FreeBase (important: dont copy and paste + cite correctly!)

\paragraph{Access methods}
\label{sec:access_methods}
% how are they accessed
% TODO S: Which access methods did we chose and why did we chose them?

\paragraph{Combination of knowledge bases}
\label{sec:dataset_combination}
% how are they combined
% TODO M: SILK
% COMMENT FROM EMAIL the proposal looks very nice! 


%In case you want to tackle the interlinking via SILK, it would be an asset in itself to publish a linkset between ProductDB and DBpedia. 

%Along the same lines, since interlinking can be very well evaluated using recall, precision, and F-measure, I'd like to ask you to produce a small gold standard (say: 100 instances plus their interlinks), and then evaluate how well your interlinking works.

\chapter{Techniques Used}
\label{cha:technique}

%TODO: add paragraphs for the techniques we are using
\paragraph{TODO}
\label{sec:TODO}


\chapter{Example results}
\label{cha:example}

\paragraph{Outcome}
\label{sec:outcome}
% What outcomes does the application provide?
% TODO: what is meant by "outcome"?

\paragraph{Queries}
\label{sec:query}


% TODO: Aspirin (drug from Bayer) among them Ecotrin (-> finds 2 manufacturers for Ecotrin + generics)
% TODO: Lamotrigine (active substance)

\label{ex:ecotrin}


\chapter{Known limitations}
\label{cha:limiations}


\paragraph{Mu}
%TODO: Check if NeoCitran is part of resultset


\paragraph{Current manufacturer}
% TODO: formulate
FreeBase does not consider time dimension. Therefore multiple drug manufacturers can be found for the same brand if the brand was e.g. sold. An example is Ecotrin (-> page \pageref{ex:ecotrin}) which was developed and produced by GlaxoSmithKline. In 2011 it was sold in bundle deal to Prestige Brands, which is its current manufacturer.
% TODO: formulate
Workaround: crawl information from trademark registers. Add information about production time frames to output, since drugs produced from previous brand owners could still be queried.

\paragraph{Quantity of active substance}
% TODO: formulate
DBPedia does refer all generics to a single entity, it does not contain information about the quantity of active substances. Therefore a 500mg Aspirin is considered the same as a 200mg Aspirin.
% TODO: formulate
Workout: filter quantity from user input and check it on corresponding FreeDB pages.



\paragraph{Impossible queries}
\label{sec:unquery}
\subparagraph{Multi ingredient drugs}No multi ingredient drugs e.g. "Neo Citran". Because DBPedia does not consider it as a SameAs relationship -> no generics found.

\subparagraph{Drugs not listen as sameAs}
% TODO: formulate
If a drug is not linked correctly on DBPedia, e.g. it has an own page, even if the ingredients are identical, the search wont function.


\paragraph{Lessons learned}
\label{cha:lessons}
%TODO: at the end


% TODO: delete?
\paragraph{Use datasets with multiple maintainers} since our initial project proposal, searching manufacturers of certain product categories and visualizing them on Google maps, had to be discarded, since ProductDB was shut down and the only maintainer was unreachable.


\paragraph{Challenges}
\label{sec:challenges}

1. SameAs 
2. Time

\paragraph{Biggest Obstacles}
\label{sec:obstacle}

%\section{More ideas} % TODO S: where does this section come from?
\label{sec:idea}

%\bibliographystyle{plain} %TODO: we might not need it.
%\bibliography{thesis-ref} %TODO: we might not need it.


\appendix

\chapter{Program Code / Resources}
\label{cha:appendix-a}

The source code can be obtained via git. Access to the private repository will be granted upon request by silviadali101@gmail.com


%\chapter{Further Experimental Results}
%\label{cha:appendix-b}

%TODO: section needed?
\newpage

\end{document}
