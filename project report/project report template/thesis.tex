% do not change these two lines (this is a hard requirement
% there is one exception: you might replace oneside by twoside in case you deliver 
% the printed version in the accordant format
\documentclass[11pt,titlepage,oneside,openany]{book}
\usepackage{times}


\usepackage{graphicx}
\usepackage{latexsym}
\usepackage{amsmath}
\usepackage{amssymb}

\usepackage{ntheorem}

% \usepackage{paralist}
\usepackage{tabularx}
\usepackage{url}

% this packaes are useful for nice algorithms
\usepackage{algorithm}
\usepackage{algorithmic}


\usepackage{caption}

% well, when your work is concerned with definitions, proposition and so on, we suggest this
% feel free to add Corrolary, Theorem or whatever you need
\newtheorem{definition}{Definition}
\newtheorem{proposition}{Proposition}


% its always useful to have some shortcuts (some are specific for algorithms
% if you do not like your formating you can change it here (instead of scanning through the whole text)
\renewcommand{\algorithmiccomment}[1]{\ensuremath{\rhd} \textit{#1}}
\def\MYCALL#1#2{{\small\textsc{#1}}(\textup{#2})}
\def\MYSET#1{\scshape{#1}}
\def\MYAND{\textbf{ and }}
\def\MYOR{\textbf{ or }}
\def\MYNOT{\textbf{ not }}
\def\MYTHROW{\textbf{ throw }}
\def\MYBREAK{\textbf{break }}
\def\MYEXCEPT#1{\scshape{#1}}
\def\MYTO{\textbf{ to }}
\def\MYNIL{\textsc{Nil}}
\def\MYUNKNOWN{ unknown }
% simple stuff (not all of this is used in this examples thesis
\def\INT{{\mathcal I}} % interpretation
\def\ONT{{\mathcal O}} % ontology
\def\SEM{{\mathcal S}} % alignment semantic
\def\ALI{{\mathcal A}} % alignment
\def\USE{{\mathcal U}} % set of unsatisfiable entities
\def\CON{{\mathcal C}} % conflict set
\def\DIA{\Delta} % diagnosis
% mups and mips
\def\MUP{{\mathcal M}} % ontology
\def\MIP{{\mathcal M}} % ontology
% distributed and local entities
\newcommand{\cc}[2]{\mathit{#1}\hspace{-1pt} \# \hspace{-1pt} \mathit{#2}}
\newcommand{\cx}[1]{\mathit{#1}}
% complex stuff
\def\MER#1#2#3#4{#1 \cup_{#3}^{#2} #4} % merged ontology
\def\MUPALL#1#2#3#4#5{\textit{MUPS}_{#1}\left(#2, #3, #4, #5\right)} % the set of all mups for some concept
\def\MIPALL#1#2{\textit{MIPS}_{#1}\left(#2\right)} % the set of all mips


\begin{document}

\pagenumbering{roman}
% lets go for the title page, something like this should be okay
\begin{titlepage}
	\vspace*{2cm}
  \begin{center}
   {\Huge Finding Generic Drugs \\}
	\vspace*{0.5cm}
   {\large  interlinking DBPedia with FreeDB based on drug brands \\}
   \vspace{2cm} 
   \vspace{2cm}
   {written by\\
    Markus Dietsche 1513384 \\
    Dandan Li 1486051\\
   }
   \vspace{1cm} 
   {submitted to the\\
    Data and Web Science Group \\
    Prof.\ Dr.\ Paulheim\\
    University of Mannheim\\} \vspace{2cm}
   {December 2015}
  \end{center}
\end{titlepage} 

% no lets make some add some table of contents
\tableofcontents
\newpage

%\listofalgorithms %TODO: we dont need it

%\listoffigures %TODO: we dont need it

%\listoftables %TODO: we dont need it

% evntuelly you might add something like this
% \listtheorems{definition}
% \listtheorems{proposition}

\newpage


% okay, start new numbering ... here is where it really starts
\pagenumbering{arabic}

% Legend:
% TODO: 	general todo, doesnt matter who takes care of it
% TODO M: 	todo for Markus
% TODO S: 	todo for Silvia


\chapter{Application domain and goals}
\label{cha:domain}
This documents describes the development of a generics search engine, based on the publicly available knowledge bases DBPedia\footnote{\url{www.dbpedia.org}} and FreeBase\footnote{\url{www.freebase.com}}. Furthermore it discusses the limitations of the corresponding TRL6\footnote{Technology Readiness Level \url{http://resources.sei.cmu.edu/library/asset-view.cfm?assetID=5835}} prototype and contains proposals for further development.

\paragraph{Problem} \label{problem}
% Which user information needs are addressed
Currently people are forced to rely on expert opinion to find a substitute, e.g a cheaper alternative, to their medication. Since many experts, like doctors or pharmacists, are granted benefits from pharmaceutical companies their judgment is often assorted to be biased.
Furthermore drugs often have fantasy names, due to marketing reasons. This is in particular the case for over-the-counter drugs. Therefore a comparison is only possible based ingredients.
In case of drugs without patent protection this potentially results in filtering thousands of different drugs by their ingredients. Needless to say that even well informed an unbiased expert will not know all generics by heart.


\paragraph{Solution} 
%\paragraph{Target group}\label{users}
% Which users are targeted?
The search engine is designed for private individuals, without any medical expert knowledge.
\label{solution} % Which user problems are solved?
 By entering a drug brand name, ATC code\footnote{Anatomical Therapeutic Chemical (ATC) Classification System} or active substance the prototype will return a list of identical substitute drugs and their corresponding drug manufacturers.


\chapter{Datasets}
\label{cha:datasets}

\paragraph{Datasets used}
\label{datasets_used}
% which datasets does the application use
The application accesses drug related data from DBPedia (DB) and FreeBase (FB).

\subparagraph{DBPedia vs. FreeBase}
\label{db_vs_fb}
Both DB and FB are based on information extracted from Wikipedia and the schema mappings can be edited. Yet FB imported data from a wide variety of additional resources and was user-editable, while DB can only be edited by editing Wikipedia content. Another difference is that DB uses regular RDF triplets to store data, while FB uses n-tuples.\footnote{\url{http://wiki.freebase.com/wiki/DBPedia}}.

\subparagraph{DBPedia}
\label{dbpedia}

DB considers all drug brands for the same active substance as the same entity. It contains a \textit{dbo:wikiPageRedirects\footnote{\url{http://dbpedia.org/ontology/wikiPageRedirects}}} for each drug brand, active substance or ATC code. The extraction of the brand names is explained on page \pageref{limitations:filtering}.


\subparagraph{FreeBase}
\label{freebase}

FB holds additional information about drug brands in two entities. For some  drugs it contains a \textit{drug\_brand}, which contains \textit{manufactured products} of the brand (e.g. Ecotrin 325 coated tablet is a manufactured product of the \textit{drug\_brand} Ecotrin. Additionally the \textit{manufacturer} and \textit{website} is stored for \textit{drug\_brand} and \textit{manufactured products}.
For the drug brands where no additional information beside the label is contained, it is stored as \textit{alias} in a \textit{drug} entity ($\rightarrow$ p. \pageref{limitations:fb_drugs}).  A variety of issues due to the discontinuance of FB is discussed on page \pageref{prob:freebase}.



\paragraph{Access methods}
\label{access_methods}
In consequence of the the endorsed accessing ways SPARQL ($\rightarrow$ p. \pageref{sparql}) is used for FB and MQL ($\rightarrow$ p. \pageref{mql}) for FB. Both are accessed online v9a the official SPARQL endpoint\footnote{\url{http://dbpedia.org/sparql}} of DB and the official MQL google web API\footnote{\url{https://www.googleapis.com/freebase/v1/mqlread}} for freebase. Accessing FB via SPARQL is currently sub-optimal, the reason for this is discussed on page \pageref{challenges:fb}.



\paragraph{Combination of knowledge bases}
\label{dataset_combination}
% how are they combined
The two knowledge bases are combined in a 3 step process: I. DBpedia gathering of labels, II. Local filtering of labels and III. FreeDB validation and completion.


\subparagraph{I. DBPedia} contains information about which drug brands exist by considering them as the same entity as the active substance and redirecting them with \textit{dbo:wikiPageRedirect}. The case the active substance is the input, a list of redirects can be obtained directly. In case a brand name is searched, the redirect has to be followed to the corresponding active substance entity to obtain the list.

\subparagraph{II. Filtering}
The list contains labels for different brands of the same drug, ATC codes and chemical formulas. Fortunately ATC code label contain \_code\_ and (most) formulas end with \_acid. Therefore all list items with match the corresponding regular expression can be deleted. Still some ATC codes and formulas might pass the current regex ($\rightarrow$ p. \pageref{limitations:filtering}).

\subparagraph{III. FreeDB}
The filtered list is used as input for FreeBase. The FB queries can be seperated in two cases: In the first case the search term is a \textit{drug\_brand} and FB has information about \textit{manufacturer}, which is the brand holder, a list of \textit{products} and the \textit{website} of the manufacturer. Afterwards new queries for each item on the products list, will return the corresponding \textit{manufacturer} of the specific product and the products \textit{website}. In the second case FB has no separate entity for a search term, it might still has it as a alias for a \text{drug} to verify it´s a drug brand.


\chapter{Techniques Used}
\label{cha:technique}
%TODO: add paragraphs for the techniques we are using



The techniques applied in our project will be discussed in this section. The viable prototype, which was also used for the field trial ($\rightarrow$ p. \pageref{field_trial}), is available at \url{http://generics.markus-dietsche.de}.

\paragraph{Prototype}
\label{prototype}


To ensure a user-friendly field trial, a web application was developed. It's based on 
\textbf{Django}\footnote{\url{https://www.djangoproject.com/}}, a framework for web applications written in \textbf{Python} and \textbf{Bootstrap}\footnote{\url{https://getbootstrap.com/}} a web frontend framework.

\label{sparql}
\textbf{SPARQL}, a query language to access Resource Description Framework (RDF) data, is used to access DBPedia via it's SPARQL endpoint. On client side \textbf{SPARQLWrapper}\footnote{\url{https://rdflib.github.io/sparqlwrapper/}} is used to facilitate the query creation and result converting. 

\label{mql}
Metaweb Query Language \textbf{MQL}\footnote{\url{https://developers.google.com/freebase/v1/mql-overview}}, is the endorsed way to access information from FreeBase. MQL was chosen since the official SPARQL endpoint\footnote{\url{http://www.freebase.com/base/sparql}} of FreeBase is offline, the FactForge endpoint\footnote{\url{http://factforge.net/sparql}} is unreliable and slow and a local installation would consume with 250GB\footnote{\url{https://developers.google.com/freebase/data}} too much space for testing purposes.

Since the discontinuance of freebase ($\rightarrow$ p. \pageref{prob:freebase}) registration is impossible. Unfortunately the official freebase python wrapper \footnote{\url{https://code.google.com/p/freebase-python/}} requires a user account and can therefore not be used. Due to this a homegrown MQL wrapper, to access the web MQL client, was written.


\chapter{Evaluation}
\label{cha:evaluation}

\paragraph{Example results}

To replicate the example results, please enter the corresponding search term on \url{http://generics.markus-dietsche.de}. Due to the size of most result sets only extracts will be discussed in this section.

\begin{center}
\begin{tabular}{rccll}
\hline 
 & \textbf{Product Name} &\textbf{ Manufacturer}  \\ 
\hline 
1 & Altruline &  \\ 
\hline 
2 & Apo-Sertraline & \\ 
\hline 
3 & Asentra &  \\ 
\hline 
4 & Atruline	 &  \\ 
\hline 
5 & Besitran	 &  \\ 
\hline 
6 & C17H17Cl2N &  \\ 
\hline 
...  &  &  \\ 
\hline 
30 & Zoloft &	Pfizer \\ 
\hline 
31 & Zoloft 50 film coated tablet & Roerig, Inc \\ 
\hline 
32 & Zoloft 25 film coated tablet & Roerig, Inc \\ 
\hline 
33 & Zoloft 50 film coated tablet & Roerig, Inc \\ 
\hline 
34 & Zoloft 100 film coated tablet & Roerig, Inc \\ 
\hline 
35 & Zoloft 20 concentrate solution & Roerig, Inc \\ 
\hline 
36 & Zoloft 100 film coated tablet & Roerig, Inc \\ 
\hline 
\end{tabular}
\captionof{table}{Extract of results for "Lustral"} 
\label{example:lustral}
\end{center}

For the Lustral query the result set contains a drug brand with a corresponding owner (Nr. 30, Zoloft, is owned by Pfizer), yet the product itself is manufactured from another company (Roerig, Inc). Furthermore 5 drug brands (Nr. 1 until Nr. 5) without any further information ($\rightarrow$ p. \pageref{limitations:fb_drugs}), a chemical formula with incorrect syntax (Nr. 6, C17H17Cl2N) ($\rightarrow$ p. \pageref{limitations:filtering}) and duplicate products ($\rightarrow$ p. \pageref{limitations:dubplicates}).

\begin{center}
\begin{tabular}{rccll}
\hline 
 & \textbf{Product Name} &\textbf{ Manufacturer}  \\ 
\hline 
1 & Concerta &  \\ 
\hline 
...  &  &  \\ 
\hline 
28	&Methylin	&\\ 
\hline  
29	&Methylin 5 solution	&Shionogi \\ 
\hline  
30	&Methylin 10 solution	&Shionogi \\ 
\hline 
31	&Methylin 10 tablet	&Mallinckrodt \\ 
\hline 
32	&Methylin 20 tablet	&Mallinckrodt \\ 
\hline 
33	&Methylin 5 tablet	&Mallinckrodt \\ 
\hline 
34	&Methylin 5 chewable tablet	&Shionogi \\ 
\hline 
35	&Methylin 2.5 chewable tablet	&Shionogi \\ 
\hline 
36	&Methylin 10 tablet	&Mallinckrodt \\ 
\hline 
37	&Methylin 5 tablet	&Mallinckrodt \\ 
\hline 
38	&Methylin 20 tablet	&Mallinckrodt \\ 
\hline 
39	&Methylin 10 chewable tablet	&Shionogi\\ 
\hline 
...  &  &  \\ 
\hline  

67 & Ritaline &  \\ 
\hline 
\end{tabular}
\captionof{table}{Extract of results for "Ritalin"} 
\label{example:ritalin}
\end{center}

For the Ritalin query the corresponding result set contains an example for a drug brand (Nr. 28 Methylin) which uses multiple manufacturers for it's products: Shionogi for solutions and chewable tablets and Mallinckrodt for regular tablets. Furthermore it contains the french name of Ritalin brand (Nr. 67, Ritaline).

\paragraph{Field Trial}
\label{cha:field_trial}
\label{field_trial}
A field trial with 35 testers was conducted from 6th until the 8th of December 2015.

Thanks to the support of Dr. Diana Bursan (Iuliu Hațieganu University of Medicine and Pharmacy) 11 Romanian medical experts delivered feedback from a professional point of view (group A). This is in particular important because the target user group described on page \pageref{cha:domain} can not rate the quality and completeness of the results.
In addition 24 random non-experts (group B) from Germany, Romania and Canada participated and represented the target user group, their feedback will be discussed in the outcome section on page \pageref{outcome}. 

A total of 309 queries\footnote{Queries are conducted anonymously, so there is no information about which group conducted them.} was analyzed. 184 queries returned results (59.5\%), 102 were misspelled (33.0\%), which mostly was due to the first letter being lowercase or the brand name in foreign languages , and 23 (7.5\%) did not return any results.  

\paragraph{Quality of knowledge base linkage}
\label{cha:gold standard}
A gold standard of 100 instances are produced to assess the quality of interlinking between DBPedia and Freebase. Among 100 tests, 2079 alternative brand drugs are obtained with 26\% filtered by combination requirement. 101 manufacturers corresponding to brand drug are extracted while 476 manufactured products got correct manufacturer information to user.  Finally,31 out of 101 manufacturer information are correctly acquired.
\begin{table}[h]
\centering
\label{my-label}
\begin{tabular}{lllll}
                          &       & \multicolumn{2}{l}{Search Engine} & Precision \\\hline
Accuracy:94\%             &       & True            & False           &       \\\hline
Freebase                  & True  & 31              & 52              & 37.3\%    \\\hline
                          & False & 70              & 1923            & 96.5\%    \\\hline
Recall                    &       & 30.1\%          & 97.4\%          &          
\end{tabular}
\captionof{table}{gold standard evaluation}
\end{table}
\paragraph{Outcome}
\label{outcome}
% What outcomes does the application provide?
% TODO: what is meant by "outcome"?
To quantify the outcome a field trial has been conducted ($\rightarrow$ p. \pageref{cha:field_trial}). Group (A) consists of medical professionals and (B) the target user group.

(A) medical experts: stated that the majority of brands in the result set are from the North American market and many European brands are not contained. Furthermore many brands have different names in different languages, in the fewest cases those ones are considered.

\begin{center}
\textit{"Very convenient while working abroad."}\linebreak
- Dr. Dalia Onetiu, Romanian doctor working in France
\end{center}

(B) target user group: The biggest obstacle for the target user group is the correct spelling of the sometimes unusual drug names. Furthermore they tend to search for over-the-counter drug cocktails (e.g. Neo Citran) which are not considered in the search at all.

\paragraph{Conclusion}
\label{conclusion}

One obstacles remains for the non-expert target user group: 
A large percentage of over-the-counter drugs often contain multiple active substances (e.g. a pain killer + caffeine is a common combination). Those ones can not be found by this approach, since DBPedia only considers two drugs to be the same when they contain the the same substance and no other active substances.

Based on the field study and the analyzed queries the results cater more to a professional user group than initially expected. Therefore the rational conclusion is to change the focus to an medical expert target group and cater to their need. The next logical features for this user group have been extracted from the field study feedback: information about the active substance and it's side effect, as well as information about the country in which a drug brand is sold.


% TODO: Aspirin (drug from Bayer) among them Ecotrin ($\rightarrow$ finds 2 manufacturers for Ecotrin + generics)
% TODO: Lamotrigine (active substance)

\label{ex:ecotrin}


\chapter{Known limitations}
\label{cha:limiations}



\paragraph{FreeBase}
\label{prob:freebase}
Google decided to discontinue FreeBase in June 2015\footnote{\url{https://plus.sandbox.google.com/109936836907132434202/posts/bu3z2wVqcQc}} to support the growth of Wikidata, which does currently not contain the information this project requires. Due to this the project switched into read-only mode and the knowledge will not further expand. The only possibility for the search engine to improve is if DBPedia adds new redirects for which FreeDB already contains information. 


\paragraph{Aliases vs. drug brands}
\label{limitations:fb_drugs} 
In case that a drug brand is only known due to it's \textit{alias} on FB, it holds the same information as it's corresponding \textit{wikiPageRedirects} from DB. Due to this additional information about the \textit{manufacturer} or \textit{website} can be obtained.


\paragraph{Filtering}
\label{limitations:filtering}
In some cases DBPedia redirects form syntactically incorrect chemical formulas (see Lustral example on $\rightarrow$ p. \pageref{example:lustral}) and APC codes. \textbf{Workaround} A more elaborated regular expression for filtering.

\paragraph{Ownership over time}
% TODO: formulate
FreeBase does not consider the time dimension for ownership. Therefore multiple drug manufacturers can be found for the same brand if the brand was e.g. sold. An example is Ecotrin ($\rightarrow$ p. \pageref{ex:ecotrin}) which was developed and produced by GlaxoSmithKline. In 2011 it was sold in bundle deal to Prestige Brands, which is its current manufacturer.
% TODO: formulate
\textbf{Workaround} crawl information from trademark registers. Add information about production time frames to output, since drugs produced from previous brand owners could still be queried.

Furthermore for manufactured products a manufacturer change is sometimes contained by an additional entity, which only has a different manufacturer as difference. See \textbf{Dubplicate products} for more details

\paragraph{Quantity of active substance}
% TODO: formulate
DBPedia does refer all generics to a single entity, it does not contain information about the quantity of active substances. Therefore a 500mg Aspirin is considered the same as a 200mg Aspirin.
% TODO: formulate
\textbf{Workaround} filter quantity from user input and check it on corresponding FreeDB pages.


\paragraph{Duplicate products}
\label{limitations:dubplicates} 
The current MQL query receives the manufacturer due to the label of the manufactured product. Due to this only one manufacturer is shown in the result set for entities with the same label. An exmaple is \textit{Methylin 10 tablet}, from out Ritalin example on  p. \pageref{example:ritalin}, which has two entities in freebase \url{http://www.freebase.com/m/0hqprm_} manufactured by Mallinckrodt and \url{http://www.freebase.com/m/0hqh8ty} manufactured by Golden State Medical Supply. \textbf{Remark} there are dates stored for the start and end of related marketing campaign. Based on this an assumption could be made which one is the current manufacturer, yet this would not be reasonable proof. \textbf{Workaround} a more elaborated query which follows the links of the corresponding entities, instead of withdrawing a manufacturer for the label of the manufactured product.


\paragraph{Impossible queries}
\label{drug_cocktail}
\subparagraph{Multi ingredient drugs} No multi-ingredient drugs e.g. "Neo Citran" can be found with this approach. This is due to the fact that DBPedia does not consider it as a wikiPageRedirect relationship $\rightarrow$. this is only the case for pure active substances.

\subparagraph{Drugs brands in differnt languages}
\label{drug_foregin_name}
Many drug brands use different names in different languages, e.g. Aspirin in English, Aspirine in French and Aspirin\u{a} in Romanian. Since the Romanian spelling is not known to DBPedia it cant be searched. This is apparently the case for many other languages and drugs. \textbf{Workaround} the search term could be send to an online dictionary, which is able to identify languages e.g. Google translate.
% TODO: formulate

\subparagraph{Spelling}
\label{drug_spellcheck}
The field study has shown that the target user group of non-expert has increased problems in spelling drug names correctly. \textbf{Workaround} connecting the search engine with an online spell check e.g. the Google spell check API.



\paragraph{Lessons learned \& Challenges overcame}
\label{cha:lessons}

\label{challenges}

\subparagraph{Knowledge base selection} Our initial project proposal, searching manufacturers of certain product categories and visualizing them on Google maps, had to be discarded, since ProductDB was shut down, no dump is available and the only known maintainer is unreachable. Furthermore the usage of FreeBase is sub-optimal since it is discontinued. Yet we did not find an alternative source for the drug details we required. 


\subparagraph{FreeBase access}
\label{challenges:fb}
To gain stable access to FreeBase (FB) was harder than expected. Since the official SPARQL endpoint is offline, FactForge was initially chosen, as the only publicly available replacement. Yet it turned out that it's unreliably slow and occasionally unreachable. Since the data dump has a uncompressed size of 250 GB a local instance was not an option. 
So we decided to use the endorsed way to access FB: MQL. After the queries have been rewritten and tested in the FB online MQL client it turned out that the official MQL wrapper requires a FB account, and the registration is closed for good.
Luckily there is an MQL web API from google, which returns the results as JSON. Based on this we developed a homebrewn wrapper for the web API.



\appendix
%\chapter{Further Experimental Results}
%\label{cha:appendix-b}

%TODO: section needed?
\newpage

\end{document}
